\documentclass{article}
\usepackage[utf8]{inputenc}
\usepackage[margin=1.0in]{geometry}

\title{Search and (Bayesian) Rescue: Using statistics to locate inobservable planets}
\author{GM, DD, MLD}

\begin{document}

\maketitle

\section{Abstract} Help! In an effort to escape the COVID-19 pandemic, your friends have decided to move to remote planets within our solar system that can no longer be observed. However, they got lost on their way. You must determine all of the planetary masses correctly and determine how many planets are involved to help your friends find their way. Can you?

\section{Proposal}

Planet detection based on $N + m$ body simulations. Based on observed planet trajectories and a known set of $N$ planets can we use our knowledge of data analysis and physics to determine whether the system includes $m$ additional planets, and to what level of credibility we think this is true.

\textbf{Big naive goal:} Predict (with confidence) that Planet 9 exists.

\textbf{More reasonable goal:} Statistically prove that a 2-body orbit is unrealistic without the presence of a 3rd body.

\textbf{Implementation:} Python simulations and Bayesian MCMC analysis. The simulations will be performed using (Hamiltonian mechanics / basic MD integration of the equations of motions) to model the motion of the planets. The initial conditions (phase space) will be specified for the N bodies along with their masses. 

To determine how many additional planets will be involved, we will apply model selection where $m$ additional planets are involved into the $N$ body simulation. Each of the additional planets will have a variable $M_i$ that allows for the planet's mass to be Bayesian inferred. We could also have Bayesian parameters for the initial conditions of the additional $m$ masses, potentially, or have them fixed. Then, using the principles of model selection, we can determine what is the best model to predict the orbital motion of the $N$ observed planets by determining when the $M_i$ are sufficiently close to zero such that the planets have been ruled out of the simulations. Then, after completing the MCMC, all the $M_i$ that are sufficiently large will indicate that the presence of these additional bodies is important to model the trajectory of the $N$ original bodies. We should extract with what credibility regions we can determine the mass of these additional $m$ bodies.

\textbf{Animations:} Because scientific education is important and is greatly enhanced by visuals, we would also like to animate the motion of the planets and have the $M_i$ parameters correspond to the opacity of the additional bodies. We can overlay the actual planet trajectories and those caused by the additional $m$ bodies. This is more driven by our personal interest in the computations involved, and the potential benefit to clear communication of our results.

\textbf{Being good students:} To make the project go as planned, we will make sure to commit early and often, and also to write unit tests for as many functions as we can. The work load will be evenly distributed between all members of this team which will also contribute to all parts of this project.

\textbf{Signatures:} Michael Lindner-D'Addario, Delaney Dunne, and Gabriella Morin.

\section{Milestones}

\begin{enumerate}
    \item N-body simulation (Cartesian coordinates, initial conditions fixed): April 5, 2020, DD
        \subitem model motion
        \subitem get it running
        \subitem (bonus) animation
   
    \item Likelihood, prior, posterior function (masses of invisible planets as only MCMC parameters): April 8, 2020, MLD
        \subitem Gaussian errors?
        \subitem Only the final time step?
        \subitem The final 10\% of the time steps?

    
    \item Examples for paper results: April 11, 2020, GM
        \subitem Binary system ($\sim$ equal masses)
            \subsubitem 2 + 0
            \subsubitem 2 + 1
            \subsubitem 2 + 2
        \subitem Solar system ($\sim$ big mass frame)
            \subsubitem 2 + 0
            \subsubitem 2 + 1
            \subsubitem 2 + 2
    \item Paper writing: April 14, 2020, GM, DD and MLD (exact contribution TBD, depending on previous contribution weight)
        \subitem Intro
        \subitem Body
        \subitem Conclusion

    \item\textbf{Project Due:} April 14, 2020

\end{enumerate}


    \section{Physical Background}

    Simple deterministic Hamiltonian simulation based on the interaction between pairs of gravitationally bound bodies:

    \begin{equation}\label{eq:Fij}
        \vec{F}_{ij} = \frac{G m_i m_j}{|| \vec{q}_i - \vec{q}_j} (\vec{q}_i - \vec{q}_j)
    \end{equation}

    is the force acting on one body with with $m_i$, position $\vec{q}_i$ from another body with mass $m_i$ and position
    $\vec{q}_j$.

    Summing over all contributions, the equation of motion for a single body is therefore

    \begin{equation}\label{eq:EoMofbodyi}
        m_i \frac{\mathrm{d}^2 \vec{q}_i}{\mathrm{d} t^2} =
        \sum^n_{j=1, j\neq i} \frac{G m_i m_j}{|| \vec{q}_i - \vec{q}_j} (\vec{q}_i - \vec{q}_j).
    \end{equation}

    Integrating the acceleration over the required time interval yields the position and velocity of the body after
    that time has passed.

\end{document}
